\documentclass[]{article}
\usepackage{lmodern}
\usepackage{amssymb,amsmath}
\usepackage{ifxetex,ifluatex}
\usepackage{fixltx2e} % provides \textsubscript
\ifnum 0\ifxetex 1\fi\ifluatex 1\fi=0 % if pdftex
  \usepackage[T1]{fontenc}
  \usepackage[utf8]{inputenc}
\else % if luatex or xelatex
  \ifxetex
    \usepackage{mathspec}
  \else
    \usepackage{fontspec}
  \fi
  \defaultfontfeatures{Ligatures=TeX,Scale=MatchLowercase}
\fi
% use upquote if available, for straight quotes in verbatim environments
\IfFileExists{upquote.sty}{\usepackage{upquote}}{}
% use microtype if available
\IfFileExists{microtype.sty}{%
\usepackage[]{microtype}
\UseMicrotypeSet[protrusion]{basicmath} % disable protrusion for tt fonts
}{}
\PassOptionsToPackage{hyphens}{url} % url is loaded by hyperref
\usepackage[unicode=true]{hyperref}
\hypersetup{
            pdftitle={R Notebook},
            pdfborder={0 0 0},
            breaklinks=true}
\urlstyle{same}  % don't use monospace font for urls
\usepackage[margin=1in]{geometry}
\usepackage{color}
\usepackage{fancyvrb}
\newcommand{\VerbBar}{|}
\newcommand{\VERB}{\Verb[commandchars=\\\{\}]}
\DefineVerbatimEnvironment{Highlighting}{Verbatim}{commandchars=\\\{\}}
% Add ',fontsize=\small' for more characters per line
\usepackage{framed}
\definecolor{shadecolor}{RGB}{248,248,248}
\newenvironment{Shaded}{\begin{snugshade}}{\end{snugshade}}
\newcommand{\KeywordTok}[1]{\textcolor[rgb]{0.13,0.29,0.53}{\textbf{#1}}}
\newcommand{\DataTypeTok}[1]{\textcolor[rgb]{0.13,0.29,0.53}{#1}}
\newcommand{\DecValTok}[1]{\textcolor[rgb]{0.00,0.00,0.81}{#1}}
\newcommand{\BaseNTok}[1]{\textcolor[rgb]{0.00,0.00,0.81}{#1}}
\newcommand{\FloatTok}[1]{\textcolor[rgb]{0.00,0.00,0.81}{#1}}
\newcommand{\ConstantTok}[1]{\textcolor[rgb]{0.00,0.00,0.00}{#1}}
\newcommand{\CharTok}[1]{\textcolor[rgb]{0.31,0.60,0.02}{#1}}
\newcommand{\SpecialCharTok}[1]{\textcolor[rgb]{0.00,0.00,0.00}{#1}}
\newcommand{\StringTok}[1]{\textcolor[rgb]{0.31,0.60,0.02}{#1}}
\newcommand{\VerbatimStringTok}[1]{\textcolor[rgb]{0.31,0.60,0.02}{#1}}
\newcommand{\SpecialStringTok}[1]{\textcolor[rgb]{0.31,0.60,0.02}{#1}}
\newcommand{\ImportTok}[1]{#1}
\newcommand{\CommentTok}[1]{\textcolor[rgb]{0.56,0.35,0.01}{\textit{#1}}}
\newcommand{\DocumentationTok}[1]{\textcolor[rgb]{0.56,0.35,0.01}{\textbf{\textit{#1}}}}
\newcommand{\AnnotationTok}[1]{\textcolor[rgb]{0.56,0.35,0.01}{\textbf{\textit{#1}}}}
\newcommand{\CommentVarTok}[1]{\textcolor[rgb]{0.56,0.35,0.01}{\textbf{\textit{#1}}}}
\newcommand{\OtherTok}[1]{\textcolor[rgb]{0.56,0.35,0.01}{#1}}
\newcommand{\FunctionTok}[1]{\textcolor[rgb]{0.00,0.00,0.00}{#1}}
\newcommand{\VariableTok}[1]{\textcolor[rgb]{0.00,0.00,0.00}{#1}}
\newcommand{\ControlFlowTok}[1]{\textcolor[rgb]{0.13,0.29,0.53}{\textbf{#1}}}
\newcommand{\OperatorTok}[1]{\textcolor[rgb]{0.81,0.36,0.00}{\textbf{#1}}}
\newcommand{\BuiltInTok}[1]{#1}
\newcommand{\ExtensionTok}[1]{#1}
\newcommand{\PreprocessorTok}[1]{\textcolor[rgb]{0.56,0.35,0.01}{\textit{#1}}}
\newcommand{\AttributeTok}[1]{\textcolor[rgb]{0.77,0.63,0.00}{#1}}
\newcommand{\RegionMarkerTok}[1]{#1}
\newcommand{\InformationTok}[1]{\textcolor[rgb]{0.56,0.35,0.01}{\textbf{\textit{#1}}}}
\newcommand{\WarningTok}[1]{\textcolor[rgb]{0.56,0.35,0.01}{\textbf{\textit{#1}}}}
\newcommand{\AlertTok}[1]{\textcolor[rgb]{0.94,0.16,0.16}{#1}}
\newcommand{\ErrorTok}[1]{\textcolor[rgb]{0.64,0.00,0.00}{\textbf{#1}}}
\newcommand{\NormalTok}[1]{#1}
\usepackage{graphicx,grffile}
\makeatletter
\def\maxwidth{\ifdim\Gin@nat@width>\linewidth\linewidth\else\Gin@nat@width\fi}
\def\maxheight{\ifdim\Gin@nat@height>\textheight\textheight\else\Gin@nat@height\fi}
\makeatother
% Scale images if necessary, so that they will not overflow the page
% margins by default, and it is still possible to overwrite the defaults
% using explicit options in \includegraphics[width, height, ...]{}
\setkeys{Gin}{width=\maxwidth,height=\maxheight,keepaspectratio}
\IfFileExists{parskip.sty}{%
\usepackage{parskip}
}{% else
\setlength{\parindent}{0pt}
\setlength{\parskip}{6pt plus 2pt minus 1pt}
}
\setlength{\emergencystretch}{3em}  % prevent overfull lines
\providecommand{\tightlist}{%
  \setlength{\itemsep}{0pt}\setlength{\parskip}{0pt}}
\setcounter{secnumdepth}{0}
% Redefines (sub)paragraphs to behave more like sections
\ifx\paragraph\undefined\else
\let\oldparagraph\paragraph
\renewcommand{\paragraph}[1]{\oldparagraph{#1}\mbox{}}
\fi
\ifx\subparagraph\undefined\else
\let\oldsubparagraph\subparagraph
\renewcommand{\subparagraph}[1]{\oldsubparagraph{#1}\mbox{}}
\fi

% set default figure placement to htbp
\makeatletter
\def\fps@figure{htbp}
\makeatother


\title{R Notebook}
\author{}
\date{\vspace{-2.5em}}

\begin{document}
\maketitle

\section{Download data}\label{download-data}

\begin{Shaded}
\begin{Highlighting}[]
\NormalTok{path=}\StringTok{'C:/indeks.csv'}
\NormalTok{f <-}\StringTok{ }\KeywordTok{read.csv2}\NormalTok{(}\DataTypeTok{file =}\NormalTok{ path, }\DataTypeTok{header =} \OtherTok{TRUE}\NormalTok{, }\DataTypeTok{encoding =} \StringTok{'UNICOD'}\NormalTok{)}
\CommentTok{#Connect library}
\KeywordTok{library}\NormalTok{ (dplyr)}
\end{Highlighting}
\end{Shaded}

\begin{verbatim}
## 
## Attaching package: 'dplyr'
\end{verbatim}

\begin{verbatim}
## The following objects are masked from 'package:stats':
## 
##     filter, lag
\end{verbatim}

\begin{verbatim}
## The following objects are masked from 'package:base':
## 
##     intersect, setdiff, setequal, union
\end{verbatim}

\begin{Shaded}
\begin{Highlighting}[]
\CommentTok{#Have a look at the data}
\KeywordTok{glimpse}\NormalTok{(f)  }
\end{Highlighting}
\end{Shaded}

\begin{verbatim}
## Rows: 98
## Columns: 6
## $ t  <int> 1, 2, 3, 4, 5, 6, 7, 8, 9, 10, 11, 12, 13, 14, 15, 16, 17, 18, 1...
## $ Y  <dbl> 0.826, 0.812, 0.844, 0.782, 1.532, 1.564, 1.657, 1.575, 1.092, 1...
## $ X1 <dbl> 1.007, 1.012, 1.009, 1.008, 1.538, 1.597, 1.607, 1.509, 1.017, 1...
## $ X2 <dbl> 3.856, 3.887, 3.896, 3.857, 5.508, 5.510, 5.459, 5.508, 6.673, 6...
## $ X3 <dbl> 51.2, 51.8, 52.1, 50.5, 45.2, 45.4, 45.6, 46.0, 48.2, 48.1, 48.6...
## $ X4 <dbl> 0.435, 0.437, 0.433, 0.438, 0.225, 0.226, 0.231, 0.224, 0.467, 0...
\end{verbatim}

\begin{Shaded}
\begin{Highlighting}[]
\KeywordTok{head}\NormalTok{(f)}
\end{Highlighting}
\end{Shaded}

\begin{verbatim}
##   t     Y    X1    X2   X3    X4
## 1 1 0.826 1.007 3.856 51.2 0.435
## 2 2 0.812 1.012 3.887 51.8 0.437
## 3 3 0.844 1.009 3.896 52.1 0.433
## 4 4 0.782 1.008 3.857 50.5 0.438
## 5 5 1.532 1.538 5.508 45.2 0.225
## 6 6 1.564 1.597 5.510 45.4 0.226
\end{verbatim}

\subsection{Висновок: кількість спостережень:98 кількість
змінних:6}\label{ux432ux438ux441ux43dux43eux432ux43eux43a-ux43aux456ux43bux44cux43aux456ux441ux442ux44c-ux441ux43fux43eux441ux442ux435ux440ux435ux436ux435ux43dux44c98-ux43aux456ux43bux44cux43aux456ux441ux442ux44c-ux437ux43cux456ux43dux43dux438ux4456}

\subsection{Histogram}\label{histogram}

\begin{Shaded}
\begin{Highlighting}[]
\KeywordTok{library}\NormalTok{(ggplot2)}
\KeywordTok{par}\NormalTok{(}\DataTypeTok{mfrow =} \KeywordTok{c}\NormalTok{(}\DecValTok{2}\NormalTok{, }\DecValTok{3}\NormalTok{))}
\KeywordTok{hist}\NormalTok{(f}\OperatorTok{$}\NormalTok{Y, }\DataTypeTok{col =} \StringTok{'dark blue'}\NormalTok{, }\DataTypeTok{main =} \StringTok{'Y'}\NormalTok{, }\DataTypeTok{xlab =} \StringTok{'Value'}\NormalTok{)}
\KeywordTok{hist}\NormalTok{(f}\OperatorTok{$}\NormalTok{X1, }\DataTypeTok{col =} \StringTok{'dark green'}\NormalTok{, }\DataTypeTok{main =} \StringTok{'X1'}\NormalTok{, }\DataTypeTok{xlab =} \StringTok{'Value'}\NormalTok{)}
\end{Highlighting}
\end{Shaded}

\includegraphics{L1_files/figure-latex/unnamed-chunk-1-1.pdf} \#\#
Висновок:Розподіл змінних нормальний, не має довгих хвостів \#\#
Box-plot

\begin{Shaded}
\begin{Highlighting}[]
\KeywordTok{par}\NormalTok{(}\DataTypeTok{mfrow =} \KeywordTok{c}\NormalTok{(}\DecValTok{1}\NormalTok{, }\DecValTok{5}\NormalTok{))}
\KeywordTok{boxplot}\NormalTok{(f}\OperatorTok{$}\NormalTok{Y)}
\KeywordTok{boxplot}\NormalTok{(f}\OperatorTok{$}\NormalTok{X1)}
\KeywordTok{boxplot}\NormalTok{(f}\OperatorTok{$}\NormalTok{X2)}
\KeywordTok{boxplot}\NormalTok{(f}\OperatorTok{$}\NormalTok{X3)}
\KeywordTok{boxplot}\NormalTok{(f}\OperatorTok{$}\NormalTok{X4)}
\end{Highlighting}
\end{Shaded}

\includegraphics{L1_files/figure-latex/unnamed-chunk-2-1.pdf}

\begin{Shaded}
\begin{Highlighting}[]
\KeywordTok{qplot}\NormalTok{(}\DataTypeTok{data =}\NormalTok{ f, }
      \DataTypeTok{x =}\NormalTok{ X3, }
      \DataTypeTok{y =}\NormalTok{ Y, }
      \DataTypeTok{geom =} \StringTok{"boxplot"}\NormalTok{)}
\end{Highlighting}
\end{Shaded}

\begin{verbatim}
## Warning: Continuous x aesthetic -- did you forget aes(group=...)?
\end{verbatim}

\includegraphics{L1_files/figure-latex/unnamed-chunk-2-2.pdf} \#\#
Висновок:Змінні У та Х4 мають невеликі викиди, змінна Х1 значні \#\#
Violin

\begin{Shaded}
\begin{Highlighting}[]
\KeywordTok{qplot}\NormalTok{(}\DataTypeTok{data =}\NormalTok{ f, }
      \DataTypeTok{x =}\NormalTok{ X1, }
      \DataTypeTok{y =}\NormalTok{ Y, }
      \DataTypeTok{geom =} \StringTok{"violin"}\NormalTok{)}
\end{Highlighting}
\end{Shaded}

\includegraphics{L1_files/figure-latex/unnamed-chunk-3-1.pdf} \#\#
Висновок:Розподіл нормальний, не має пропущенних данних \# Statistics

\begin{Shaded}
\begin{Highlighting}[]
\KeywordTok{library}\NormalTok{ (psych)}
\end{Highlighting}
\end{Shaded}

\begin{verbatim}
## 
## Attaching package: 'psych'
\end{verbatim}

\begin{verbatim}
## The following objects are masked from 'package:ggplot2':
## 
##     %+%, alpha
\end{verbatim}

\begin{Shaded}
\begin{Highlighting}[]
\KeywordTok{describe}\NormalTok{(f)}
\end{Highlighting}
\end{Shaded}

\begin{verbatim}
##    vars  n  mean    sd median trimmed   mad   min   max range  skew kurtosis
## t     1 98 49.50 28.43  49.50   49.50 36.32  1.00 98.00 97.00  0.00    -1.24
## Y     2 98  1.01  0.22   1.00    1.00  0.25  0.58  1.66  1.08  0.37     0.04
## X1    3 98  1.02  0.18   1.01    1.01  0.17  0.68  1.61  0.93  0.83     1.40
## X2    4 98  4.92  1.27   4.32    4.80  0.63  3.34  7.60  4.26  0.81    -0.91
## X3    5 98 57.00  8.08  57.10   56.97  9.34 38.20 77.00 38.80  0.19    -0.35
## X4    6 98  0.46  0.08   0.46    0.46  0.09  0.22  0.61  0.38 -0.69     0.40
##      se
## t  2.87
## Y  0.02
## X1 0.02
## X2 0.13
## X3 0.82
## X4 0.01
\end{verbatim}

\subsection{Висновок:Пропущених данних немає,всі змінні
кількісні,розраховані медіана, середне відхилення, середнє,максимум та
мінімум по кожній із
змінних}\label{ux432ux438ux441ux43dux43eux432ux43eux43aux43fux440ux43eux43fux443ux449ux435ux43dux438ux445-ux434ux430ux43dux43dux438ux445-ux43dux435ux43cux430ux454ux432ux441ux456-ux437ux43cux456ux43dux43dux456-ux43aux456ux43bux44cux43aux456ux441ux43dux456ux440ux43eux437ux440ux430ux445ux43eux432ux430ux43dux456-ux43cux435ux434ux456ux430ux43dux430-ux441ux435ux440ux435ux434ux43dux435-ux432ux456ux434ux445ux438ux43bux435ux43dux43dux44f-ux441ux435ux440ux435ux434ux43dux454ux43cux430ux43aux441ux438ux43cux443ux43c-ux442ux430-ux43cux456ux43dux456ux43cux443ux43c-ux43fux43e-ux43aux43eux436ux43dux456ux439-ux456ux437-ux437ux43cux456ux43dux43dux438ux445}

\section{Splitting the dataset into the TRAIN set and TEST
set}\label{splitting-the-dataset-into-the-train-set-and-test-set}

\begin{Shaded}
\begin{Highlighting}[]
\KeywordTok{set.seed}\NormalTok{(}\DecValTok{123}\NormalTok{)}
\KeywordTok{library}\NormalTok{(caTools)}
\NormalTok{split =}\StringTok{ }\KeywordTok{sample.split}\NormalTok{(f}\OperatorTok{$}\NormalTok{Y, }\DataTypeTok{SplitRatio =} \FloatTok{0.7}\NormalTok{)}
\NormalTok{f_train =}\StringTok{ }\KeywordTok{subset}\NormalTok{(f, split }\OperatorTok{==}\StringTok{ }\OtherTok{TRUE}\NormalTok{)}
\NormalTok{f_test =}\StringTok{ }\KeywordTok{subset}\NormalTok{(f, split }\OperatorTok{==}\StringTok{ }\OtherTok{FALSE}\NormalTok{)}
\CommentTok{#Write prepared data to the file}
\KeywordTok{write.csv2}\NormalTok{(f_train, }\DataTypeTok{file =} \StringTok{"indeks_train.csv"}\NormalTok{)}
\KeywordTok{write.csv2}\NormalTok{(f_test, }\DataTypeTok{file =} \StringTok{"indeks_test.csv"}\NormalTok{)}
\end{Highlighting}
\end{Shaded}

\end{document}
